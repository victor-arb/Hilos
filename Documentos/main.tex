
\documentclass[12pt, letter]{article}
\usepackage[utf8]{inputenc}
\usepackage[spanish,es-tabla]{babel}
%\usepackage{times},puede ser arial 
\usepackage{csquotes}
\usepackage[left=2.54cm, right=2.54cm,top=2.54cm,bottom=2.54cm]{geometry}
\renewcommand{\baselinestretch}{1.5}
\usepackage[backend=biber,style=apa]{biblatex}
\bibliography{Referencias.bib}
\usepackage{graphicx}
\usepackage{subcaption}
\usepackage[hidelinks]{hyperref}

\title{\huge{Hilos}}
\author{Victor Manuel Arbeláez Ramírez \\ Facultad de ingeniería \\ Universidad de Antioquia}
\date{}

\begin{document}\raggedright

\maketitle

\setlength{\parindent}{31pt}
Un hilo, es un medio que administra el flujo de control de datos de un programa, separándolo en tareas que son realizadas por el microprocesador de una manera más eficiente con la ayuda de sus núcleos, gracias a la tareas o procesos generados por el programa. En otras palabras, se divide en "trozos", para optimizar los tiempos de espera de los subprocesos o threads.

\setlength{\parindent}{31pt}
Los hilos, threads o subprocesos no forman parte física del procesador, algo que habitualmente es malinterpretado, los núcleos existen a nivel de hardware y los hilos a nivel de software, ayudando a los núcleos del microprocesador a ser mucho más eficaces llevando a cabo múltiples tareas simultaneamente.

\setlength{\parindent}{31pt}
En síntesis, las principales ventajas de la programación multihilo, se dividen en la capacidad de respuesta, la compartición de recursos, la economía, y el correcto uso de la arquitectura de los procesadores o microprocesadores.

%
\setlength{\parindent}{31pt}
Tomando una visión al pasado, inicialmente el significado de un hilo, como un flujo secuencial de control, se da en 1965 con el sistema operativo Berkeley Timesharing, aunque en ese momento fueron llamados procesos y no hilos e interactuaban a través de variables compartidas.

\setlength{\parindent}{31pt}
Se piensa que el lenguaje PL / 1  fue uno de los más importantes precursores en el ámbito de lo hilos, aunque no se tiene la certeza de que algún compilador de IBM pudiese hacer uso de él, sin embargo, existen registros de que se hicieron constantes pruebas en el sistema operativo MULTICS.

\setlength{\parindent}{31pt}
Después vino Unix, en la década de 1970. La noción de Unix de un "proceso", era en un sentido “pesado” ya que no podían compartir memoria, interactuaban a través de tuberías, señales, etc, convirtiéndose así en un hilo de control secuencial más un espacio de direcciones virtuales y posteriormente, se condujo a la "invención" de subprocesos, los cuales se conocieron como "livianos", en contraste con los procesos "pesados" de Unix, dándose esta distinción a partir de los años 80, cuando aparecieron los primeros "microkernels".

%
\setlength{\parindent}{31pt}
En los modelos multihilo, se encuentra las implementaciones más comunes, en donde el soporte para los hilos se da en dos principales categorías, en el nivel de usuario (ULT) o nivel del kernel (KLT).

\setlength{\parindent}{31pt}
Para establecer relaciones entre los hilos a nivel de usuario y de kernel, se encuentran los siguientes modelos: 

\begin{itemize}
\item El modelo muchos a uno, donde se asignan múltiples hilos del nivel de usuario a un hilo del nivel de kernel.
\item El modelo uno a uno, donde se asigna cada hilo de usuario a un hilo de nivel de kernel, permitiendo la ejecución múltiple de hilos en paralelo en distintos procesadores.
\item El modelo muchos a muchos donde se transmiten muchos hilos de usuario sobre un número menor o igual de hilos de kernel, siendo éste el modelo más eficiente. 
\end{itemize}

\setlength{\parindent}{31pt}
También su puede dar la combinación de ULT y KLT en algunos sistemas operativos como lo es Solaris. En este caso, el programador, puede ajustar los parámetros para buscar el mejor resultado global en la creación de hilos, así como la mayor parte de la planificación y la sincronización es llevada a cabo en el espacio de usuario donde los distintos ULT de una aplicación se relacionan con varios KLT.

%
\setlength{\parindent}{31pt}
En las aplicaciones de hilos a nivel del kernel (KLT) pura, todo el trabajo para la gestion de hilos es realizada por el Kernel excluyendo toda participación del nivel de usuario. En el área de la aplicación solo hay una interfaz de programas de aplicación para la gestión de hilos en el núcleo. Windows 2000, Linux y OS/2 utilizan este método, aunque Linux se diferencia en qeu no hace distiniciones entre los procesos e hilos.

\setlength{\parindent}{31pt}
Existe también el módulo KVM del tipo KLT, donde con el lenguaje de programación Java se proporciona una interfaz para que el programador acceda a las funciones nativas del kernel y lograr gestionar los hilos que puedan estar ejecutándose; siendo este modelo, independiente del hardware, potable e independiente del lenguaje Java.

\end{document}\raggedright