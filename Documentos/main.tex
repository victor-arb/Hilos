
\documentclass[12pt, letter]{article}
\usepackage[utf8]{inputenc}
\usepackage[spanish,es-tabla]{babel}
%\usepackage{times},puede ser arial 
\usepackage{csquotes}
\usepackage[left=2.54cm, right=2.54cm,top=2.54cm,bottom=2.54cm]{geometry}
\renewcommand{\baselinestretch}{1.5}
\usepackage[backend=biber,style=apa]{biblatex}
\bibliography{Referencias.bib}
\usepackage{graphicx}
\usepackage{subcaption}
\usepackage[hidelinks]{hyperref}

\title{\huge{Hilos}}
\author{Victor Manuel Arbeláez Ramírez \\ Facultad de ingeniería \\ Universidad de Antioquia}
\date{}

\begin{document}\raggedright

\maketitle

\setlength{\parindent}{31pt}
Un hilo, es un medio que administra el flujo de control de datos de un programa, separado en múltiples tareas llevadas a cabo por el procesador o microprocesador y de sus diferentes núcleos de una forma más eficiente, debido a las unidades mínimas de asignación, que son las tareas o procesos de un programa; además, pueden dividirse en trozos para así optimizar los tiempos de espera de cada instrucción en la cola del proceso llamados subprocesos o threads.

\setlength{\parindent}{31pt}
Los hilos, threads o subprocesos no forman parte física del procesador, algo que habitualmente es malinterpretado, los núcleos existen a nivel de hardware y los hilos a nivel de software, ayudando a los núcleos del microprocesador a ser mucho más eficaces llevando a cabo varias tareas al tiempo.

\setlength{\parindent}{31pt}
En síntesis, las principales ventajas de la programación multihilo, se dividen en la capacidad de respuesta, la compartición de recursos, la economía, y el correcto uso de la arquitectura de los procesadores o microprocesadores.

\setlength{\parindent}{31pt}
La noción de un hilo, como un flujo secuencial de control, se remonta a 1965 con el sistema operativo Berkeley Timesharing, aunque en ese momento fueron llamados procesos y no hilos e interactuaban a través de variables compartidas.

\setlength{\parindent}{31pt}
Se piensa que el lenguaje PL / 1 en el período de 1965 fue uno de los más importantes precursores en el ámbito de lo hilos, aunque no se tiene la certeza de que algún compilador de IBM pudiese hacer pleno uso de él, sin embargo, se hicieron constantes pruebas en el sistema operativo MULTICS.

\setlength{\parindent}{31pt}
Luego vino Unix, a principios de la década de 1970. La noción de Unix de un "proceso", era en un sentido “pesado” ya que no podían compartir memoria, interactuaban a través de tuberías, señales, etc, convirtiéndose así en un hilo de control secuencial más un espacio de direcciones virtuales. 

\setlength{\parindent}{31pt}
Después de un tiempo, se condujo a la "invención" de subprocesos: procesos de estilo antiguo que compartían el espacio de direcciones de un único proceso, los cuales fueron llamados "livianos", en contraste con los procesos "pesados" de Unix. Esta distinción se remonta a finales de los años setenta o principios de los ochenta, es decir, a los primeros "microkernels" (Thoth (precursor del V-kernel y QNX), Amoeba, Chorus, la familia RIG-Accent-Mach, etc.). 

\setlength{\parindent}{31pt}
En los modelos multihilo, se encuentra las implementaciones más comunes, en donde el soporte para los hilos puedes ser proporcionado en el nivel de usuario (ULT) o por parte del kernel (KLT).

\setlength{\parindent}{31pt}
Para establecer relaciones entre los hilos a nivel de usuario y de kernel, se encuentran los siguientes modelos: 

\begin{itemize}
\item El modelo muchos a uno donde se asignan múltiples hilos del nivel de usuario a un hilo del kernel.
\item El modelo uno a uno donde se asigna cada hilo de usuario a un hilo del kernel, permitiendo la ejecución múltiple de hilos en paralelo en distintos procesadores.
\item El modelo muchos a muchos donde se multiplexan muchos hilos de usuario sobre un número menor o igual de hilos de kernel, siendo éste el modelo más eficiente. 
\end{itemize}

\setlength{\parindent}{31pt}
Algunos sistemas operativos ofrecen la combinación de ULT y KLT, como Solaris. La creación de hilos, así como la mayor parte de la planificación y sincronización de los hilos de una aplicación se realiza por completo en el espacio de usuario, en donde, los múltiples ULT de una sola aplicación se asocian con varios KLT, siendo el programador quien puede ajustar estos parámetros para obtener el mejor resultado global.

\end{document}\raggedright